%%%%%%%%%%%%%%%%%%%%%%%%%%%%%绪论%%%%%%%%%%%%%%%%%%%%%%%%%%%%%%%%%%%%%%%%%%%%%%%%%%%%%%%%%%%%%%%%%%%%%%%%%
\documentclass[UTF8]{ctexart} 
\usepackage{amsmath}
\usepackage{amsfonts}
\usepackage{amssymb}
\usepackage{bm}
\pagestyle{plain}
\title{绪论}
\date{}
 \begin{document}
    \maketitle 
    ~~方程研究的是数量的等式关系.我们本书中学习的偏微分方程又称为数学物理方程,这是指某些未知量遵循的物理规律而导出的偏微分方程,
    如热的分布,流体的流动,声音的传播等.

~~什么是方程?含有未知量的等式叫做方程.

~~什么是函数方程?含有未知函数的等式是函数方程.

~~微分方程是一种特殊的函数方程,包括常微分方程(Ordinary Differential Equation,ODE)和
偏微分方程(Partial Differential Equation,PDE),含有一元未知函数及其各阶导数的等式就称为常微分方程,
其中含有未知函数的最高阶就称为方程的阶.同理,含有多元未知函数及其各阶偏导数的等式就称为偏微分方程,
其中含有的未知函数的最高阶就称为该方程的阶.

~~常微分方程是特殊的偏微分方程.偏微分方程是常微分方程的拓广.

~~综上所述,学好偏微分方程必须在数学分析,常微分方程,复变函数等的基础上学好偏微分方程.

~~下面我们介绍一些常用的记号和公式:\\
(1)
\begin{equation*}
\frac{\partial u}{\partial x}=u_x,\frac{\partial u}{\partial y}=u_y
\end{equation*}
(2)梯度:梯度是从标量场到向量场的微分映射.\\
i.二元:设\ $u=u(x,y),(x,y)\in\Omega\subseteq R^2$
\begin{equation*}
\mathrm{grad}\ u=(u_x,u_y)\triangleq\nabla u
\end{equation*}
称为函数$u$的梯度.\\
ii.三元:设$u=u(x,y,z),(x,y,z)\in\Omega\subseteq R^3$
\begin{equation*}
\mathrm{grad}\ u=(u_x,u_y,u_z)\triangleq\nabla u
\end{equation*}\\
(3)散度:散度是从向量场到标量场的微分映射.\\
i.二元:设$\ \bm F=(P(x,y),Q(x,y)):R^2\rightarrow R$,则\\
\begin{equation*}
\mathrm{div}\ \bm F=P_x+Q_y
\end{equation*}\\
称为向量函数F的散度.\\
ii.三元:设$\ \bm F=(P(x,y,z),Q(x,y,z),R(x,y,z)):R^3\rightarrow R$,则\\
\begin{equation*}
\mathrm{div}\ \bm F=P_x+Q_y+R_z
\end{equation*}\\
拉普拉斯算子:\\
设\\
\begin{equation*}
\triangle=\frac{\partial^2}{\partial x^2}+\frac{\partial^2}{\partial y^2}+\frac{\partial^2}{\partial z^2}
\end{equation*}
则\\
\begin{equation*}
\vartriangle=\mathrm{div}(\triangledown)
\end{equation*}
称为拉普拉斯算子.
(4)旋度:旋度是从向量场到张量场的微分映射.\\
i.二元: 设$\ \bm F=(P(x,y),Q(x,y)):R^2\rightarrow R^2$,则\\
\begin{equation*}
rot\ \bm F=Q_x-P_y
\end{equation*}
称为向量函数$\bm F$的旋度.
ii.三元:设$\ \bm F=(P(x,y,z),Q(x,y,z),R(x,y,z)):R^3\rightarrow R^3$,则\\
\begin{equation*}
rot\bm F=\begin{pmatrix} i & j & k \\ \partial_x & \partial_y & \partial_z \\P & Q & R \end{pmatrix}
\end{equation*}\\
称为向量函数$\bm F$的旋度.\\
(5)二维格林公式\\
~~设$\ \bm F=(P(x,y),Q(x,y)):\Omega\subseteq R^2\rightarrow R^2$,其中$\Omega$是一个开域.
$\Gamma=\partial\Omega$是$\Omega$的边界,$\Gamma$的方向是右手方向.$\bm \tau$为$\Gamma$的单位法向量.
$\bm \tau=(\cos(\tau,x),\cos(\tau,y))$,则\\
\begin{equation*}
\iint\limits_\Omega rot\bm F dxdy=\oint\limits_\Gamma \bm F \cdot\bm \tau ds 
\end{equation*}
(6)三维格林公式\\
~~设$\ \bm F=(P(x,y,z),Q(x,y,z),R(x,y,z)):\Omega\subseteq R^3\rightarrow R^3$,其中$\Omega$是一个开域.
$\Gamma=\partial\Omega$是$\Omega$的边界,$\bm n$为$\bm S$的外法矢,则\\
\begin{equation*}
\iiint\limits_\Omega \mathrm{div}\bm F dxdydz=\iint\limits_S \bm F\cdot\bm n d\sigma 
\end{equation*}
(7)二维与三维形式的统一\\
~~二维区域$\Omega$的边界$\Gamma$的切向量$\bm \tau$与外法矢$\bm n$垂直,$\bm n$是$\bm \tau$经过顺时针
旋转$-\pi/2$之后而成的.而二维旋转变换\\
\begin{equation*}
R_\theta(x,y)=(x,y)\begin{pmatrix}\cos\theta &\sin\theta \\-sin\theta & \cos\theta\end{pmatrix}
\end{equation*}
则\\
\begin{equation*}
\begin{split}\bm{n}&=(\cos(\bm{n},x),\cos(\bm{n},y))\\
&=(\cos(\bm{\tau},x),\cos(\bm{\tau},y))\begin{pmatrix}0&-1\\1&0\end{pmatrix}\\
&=(\cos(\bm{\tau},y),-\cos(\bm{\tau},x)).
\end{split}
\end{equation*}\\
令$\bm{F_1}=(Q,-P)$,则$\mathrm{div}~\bm{F}_1=rot~\bm{F}$,\\
从而\\
\begin{equation*}
\begin{split}
\iint\limits_\Omega \mathrm{div}~\bm{F_1}dxdy&=\iint\limits_\Omega rot~\bm{F}dxdy\\
&=\int\limits_\Gamma(P(x,y)\cos(\bm{\tau},x)+Q(x,y)\cos(\bm{\tau},y))ds\\
&=\int\limits_\Gamma(-P\-\cos(\bm{\tau},x)+Q(x,y)\cos(\bm{\tau},y))ds\\
&=\int\limits_\Gamma(Q(x,y)n_1+P(x,y)n_2)ds=\int\limits_\Gamma\bm{F_1}\cdot\bm{n}ds 
\end{split}
\end{equation*}\\
所以\\
\begin{equation*}
\iint\limits_\Omega \mathrm{div}~\bm{F_1}dxdy=\int\limits_\Gamma \bm{F_1}\cdot\bm{n}ds.
\end{equation*}\\
在此,我们看到二维,三维的格林公式形式变成了统一.\\
(8)斯托克斯公式与格林公式的统一\\
~~有空间曲面$\Sigma,\Gamma=\partial\Sigma$按右手螺旋法则,$\bm{F}=(P(x,y,z),Q(x,y,z),R(x,y,z))$
在$\Sigma$上,$\bm{n}$为$\Sigma$的法矢,$\bm{\tau}$为$\Gamma$的切向,$\bm{n}$与$\bm{\tau}$构成右手法则,则有
\\\begin{equation*}
\iint\limits_\Sigma rot\bm{F}\cdot\bm{n} ds=\oint\limits_\Gamma \bm{F}\cdot\bm{\tau} ds 
\end{equation*}\\
\end{document}
%%%%%%%%%%%%%%%%%%%%%%%%%%%%%%%绪论%%%%%%%%%%%%%%%%%%%%%%%%%%%%%%%%%%%%%%%%%%%%%%%%%%%%%%%%%%%%%%%%%%
%
%                      :;J7, :,                        ::;7:
%                      ,ivYi, ,                       ;LLLFS:
%                      :iv7Yi                       :7ri;j5PL
%                     ,:ivYLvr                    ,ivrrirrY2X,
%                     :;r@Wwz.7r:                :ivu@kexianli.
%                    :iL7::,:::iiirii:ii;::::,,irvF7rvvLujL7ur
%                   ri::,:,::i:iiiiiii:i:irrv177JX7rYXqZEkvv17
%                ;i:, , ::::iirrririi:i:::iiir2XXvii;L8OGJr71i
%              :,, ,,:   ,::ir@mingyi.irii:i:::j1jri7ZBOS7ivv,
%                ,::,    ::rv77iiiriii:iii:i::,rvLq@GEEk.Li
%             ,,      ,, ,:ir7ir::,:::i;ir:::i:i::rSGGYri712:
%           :::  ,v7r:: ::rrv77:, ,, ,:i7rrii:::::, ir7ri7Lri
%          ,     2OBBOi,iiir;r::        ,irriiii::,, ,iv7Luur:
%        ,,     i78MBBi,:,:::,:,  :7FSL: ,iriii:::i::,,:rLqXv::
%        :      iuMMP: :,:::,:ii;2GY7OBB0viiii:i:iii:i:::iJqL;::
%       ,     ::::i   ,,,,, ::LuBBu BBBBBErii:i:i:i:i:i:i:r77ii
%      ,       :       , ,,:::rruBZ1MBBqi, :,,,:::,::::::iiriri:
%     ,               ,,,,::::i:  @GEEK312.       ,:,, ,:::ii;i7:
%    :,       rjujLYLi   ,,:::::,:::::::::,,   ,:i,:,,,,,::i:iii
%    ::      BBBBBBBBB0,    ,,::: , ,:::::: ,      ,,,, ,,:::::::
%    i,  ,  ,8BMMBBBBBBi     ,,:,,     ,,, , ,   , , , :,::ii::i::
%    :      iZMOMOMBBM2::::::::::,,,,     ,,,,,,:,,,::::i:irr:i:::,
%    i   ,,:;u0MBMOG1L:::i::::::  ,,,::,   ,,, ::::::i:i:iirii:i:i:
%    :    ,iuUuuXUkFu7i:iii:i:::, :,:,: ::::::::i:i:::::iirr7iiri::
%    :     :rk@Yizero.i:::::, ,:ii:::::::i:::::i::,::::iirrriiiri::,
%     :      5BMBBBBBBSr:,::rv2kuii:::iii::,:i:,, , ,,:,:i@petermu.,
%          , :r50EZ8MBBBBGOBBBZP7::::i::,:::::,: :,:,::i;rrririiii::
%              :jujYY7LS0ujJL7r::,::i::,::::::::::::::iirirrrrrrr:ii:
%           ,:  :@kevensun.:,:,,,::::i:i:::::,,::::::iir;ii;7v77;ii;i,
%           ,,,     ,,:,::::::i:iiiii:i::::,, ::::iiiir@geek312.r;7:i,
%        , , ,,,:,,::::::::iiiiiiiiii:,:,:::::::::iiir;ri7vL77rrirri::
%         :,, , ::::::::i:::i:::i:i::,,,,,:,::i:i:::r;@xingjunlong.ii:::
%%%%%%%%%%%%%%%%%%%%%%%%%%%%%%%%%%%%%%%%%%%%%%%%%%%%%%%%%%%%%%%%%%%%%%%%%%%%%%%%%%%%%%%%